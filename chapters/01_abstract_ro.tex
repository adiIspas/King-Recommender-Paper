\thispagestyle{plain}

\begin{center}
	\Large \textbf{Abstract}	
\end{center}

Prezenta lucrare de disertație își propune să îmbunătățească sistemele de recomandare pentru filme cu ajutorul informației vizuale din postere. Informația vizuală din postere este extrasă sub formă de feature-uri, iar din feature-uri se creează clustere în număr egal cu numărul de genuri disponibile pentru filme în setul de date.

\vspace{5mm}
Din perspectiva implementării sistemul poate fi împărțit după cum urmează:
\begin{enumerate}
	\item Colectarea posterelor pentru filme în acord cu filmele existente în baza de date;
	\item Extragerea feature-urilor și crearea de clustere cu ajutorul lor; 
	\item Antrenarea și evaluarea modelului cu diverse tipuri de metadate.
\end{enumerate}

Implementarea este realizată în Python 3.6 și are la bază patru librării: LightFM - folosită pentru construcția modelului de recomandare; Keras - folosită pentru extragerea de feature-uri din postere; Scikit learn - folosită pentru crearea de clustere; Skopt - folosită pentru optimizarea parametrilor modelului. Aplicația a fost rulată pe un sistem dotat cu Intel i7 Quad Core, 2.60 Ghz cu 16GB RAM și cu o extensie a memoriei swap de până la 70GB.

\vspace{5mm}
Din punct de vedere al rezultatelor au fost obținute îmbunătățiri ale metricilor de precizie@k și de acuratețe. Dacă se folosesc doar posterele precizia este îmbunătățită cu $0.42\%$, iar acuratețea cu $0.32\%$. Dacă posterele se adaugă la genuri, precizia este îmbunătățită cu $0.82\%$, iar acuratețea cu $1.09\%$.

\vspace*{\fill}
