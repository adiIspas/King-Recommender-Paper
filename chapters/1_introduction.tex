\section{Motivație}

Volumul de date crește semnificativ de la an la an astfel până în 2020 se estimează că pentru fiecare persoană de pe planetă vor fi creați în fiecare secundă 1.7 MB de date, ceea ce înseamnă peste 13 milioane de GB creați în fiecare secundă în lume.

În 2018 în fiecare minut se vizionau peste 97 de mii de ore de conținut pe Netlfix. Peste 4.3 milioane de videoclipuri erau vizionate pe Youtube. Pe Spotify se ascultau 750 de mii de melodii, iar Amazon pregătea peste o mie de pachete \hyperlink{domo}{[1]}.

\vspace{5mm}

În România, Netflix pune la dispoziție 575 de filme și 208 seriale. În Regatul Unit sunt disponibile 2425 de filme și 542 de seriale, iar în Statele Unite Ale Americii sunt disponibile 2942 de filme și 629 de seriale \hyperlink{finder}{[2]}.

Amazon oferă cumpărătorilor o gamă cu un total de peste 119 milioane de produse, dintre care 44.2 milioane de cărți, 10.1 milioane de electronice sau 4.5 milioane de produse realizate manual \hyperlink{scrapehero}{[3]}.

\vspace{5mm}

În primă fază, cu cât volumul de date pus la dispoziție de o platforma este mai mare cu atât este mai mare și necesitatea unui sistem de recomandare care să vină în ajutorul utilizatorului final pentru a explora gama de produse oferită de respectiva platformă. Ulterior, acel sistem de recomandare se vrea a fi îmbunătățit astfel încât să ofere fiecărui utilizator o experiență cât mai personalizată prin care se recomande, în cazul platformelor de streaming video, conținut relevant pentru a fi consumat de utilizatorul final, sau în cazul platformelor de ecommerce, produse pe care utilizatorul ar fi dispus să le cumpere.

\vspace{5mm}

În majoritatea cazurilor sistemele de recomandare se bazeaza pe metadatele utilizatorilor, cum ar fi: regiunea, vârsta, genul, ce alte produse a accesat sau cumpărat și metadatele produselor: categoria din care face parte, ratingul acestuia. La acestea se pot adauga și alte informații precum: ce alte produse a apreciat un alt user cu profil asemanător.
\section{Obiective propuse}


\section{Actualitate}


\section{Structura lucrării}